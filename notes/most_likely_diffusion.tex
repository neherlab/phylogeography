\documentclass[aps,rmp, onecolumn]{revtex4}
%\documentclass[a4paper,10pt]{scrartcl}
%\documentclass[aps,rmp,twocolumn]{revtex4}

\usepackage[utf8]{inputenc}
\usepackage{amsmath,graphicx}
\usepackage{color}
%\usepackage{cite}

\newcommand{\bq}{\begin{equation}}
\newcommand{\eq}{\end{equation}}
\newcommand{\bn}{\begin{eqnarray}}
\newcommand{\en}{\end{eqnarray}}
\newcommand{\Richard}[1]{{\color{red}Richard: #1}}
\newcommand{\LH}{\mathcal{L}}

\begin{document}
\title{Inference of ancestral locations}
\author{Richard Neher}
\date{\today}
\maketitle
A common phylogeographic model is diffusion.
Given the sampling locations $r_i$ of the tips $i$ of the tree, the ancestral locations $r_n$ of internal nodes of the tree have a likelihood function
\begin{equation}
    \log L = -\sum_{n\neq root} \frac{(r_n - r_p)^2}{4Dt_n}
\end{equation}
where $r_p$ is the location of the parent of $n$ and $t_n$ is the length of the branch leading to node $n$.
The ancestral locations are not observed and to calculate the overall likelihood need to be integrated over.

This can be done recursively, where the likelihood of a subtree $n$ given the position of node $r_n$ is given by
\begin{equation}
    \begin{split}
        L_n(r_n) & = \prod_{c\in n}\frac{1}{4\pi D t_c}\int  d\, r_c e^{- \frac{(r_c - r_n)^2}{4Dt_c}}P(r_c) \\
        & = \prod_{c\in n} \frac{\sqrt{d_c}}{\sqrt{\pi}}\int d\, r_c e^{- d_c r_n^2 + 2d_c r_n r_c - d_c r_c^2 - a_c r_c^2 + 2 b_c r_c - c_c} \\
        & = \prod_{c\in n} \frac{\sqrt{d_c}}{\sqrt{\pi}}\int d\, r_c e^{- d_c r_n^2 + 2r_c(b_c + r_n d_c) - r_c^2 \left(d_c + a_c\right) - c_c} \\
        & = \prod_{c\in n} \frac{\sqrt{d_c}}{\sqrt{\pi}}\int d\, r_c e^{- d_c r_n^2 - (d_c + a_c)(r_c^2 - 2r_c\frac{b_c + d_c r_n}{d_c + a_c} + \frac{(b_c + d_c r_n)^2}{(d_c + a_c)^2}) + \frac{(b_c + d_c r_n)^2}{d_c + a_c} - c_c} \\
        & = \prod_{c\in n} \frac{\sqrt{d_c}}{\sqrt{\pi}}\int d\, r_c e^{- d_c r_n^2 + (d_c + a_c)(r_c - \frac{b_c + d_c r_n}{d_c + a_c})^2 + \frac{(b_c + d_c r_n)^2}{d_c + a_c} - c_c} \\
        & = \prod_{c\in n}\frac{\sqrt{d_c}}{\sqrt{a_c + d_c}} e^{- d_c r_n^2 + (b_c^2 + 2r_n b_c d_c + d_c^2 r_n^2 )/\left(d_c + a_c\right) - c_c} \\
        & = \prod_{c\in n}\frac{\sqrt{d_c}}{\sqrt{a_c + d_c}} e^{- d_c\left(1 - \frac{d_c}{a_c+d_c}\right) r_n^2 +  2\frac{b_c d_c}{d_c+a_c} r_n  - c_c + \frac{b_c^2}{d_c + a_c}} \\
    \end{split}
\end{equation}
This allows calculation of the parameters $a_n$, $b_n$, and $c_n$ of node $n$ from the children its children, though the contributions are different for terminal and non-terminal children.
For each internal child $c$, we add
\begin{equation}
    a_n += d_c \left(1-\frac{d_c}{a_c+d_c}\right) = \frac{d_c a_c}{a_c+d_c}
\end{equation}
\begin{equation}
    b_n += \frac{b_c d_c}{a_c + d_c}
\end{equation}
\begin{equation}
    c_n += c_c  + \frac{b_c^2}{d_c + a_c} + \frac{\log(d_c) - \log(a_c+d_c)}{2}
\end{equation}
If a child is a terminal node, the terms in the sum need to be replaced by
\begin{equation}
    a_n +=  d_c
\end{equation}
\begin{equation}
    b_n += d_c r_c
\end{equation}
\begin{equation}
    c_n += d_c r_c^2 - \log(2\pi/d_c)/2
\end{equation}
Note that for a single child, the variances due to uncertainty in the child and the transmission along the branch add ($a_n^{-1} = a_c^{-1} + d_c^{-1}$) and the most likely positions don't change ($b_n/a_n = b_c/a_c$).

The same propagation can be used up the tree, where we use the index $p$ for the parent of $n$ and label coefficients of the distribution that is passed up a branch with a prime.
\begin{equation}
    a'_n  =  \frac{d_n a'_p}{a'_p + d_n} + \sum_{c\in p, c\neq n} \frac{d_c a_c}{a_c+d_c}
\end{equation}
\begin{equation}
    b'_n = \frac{b'_p d_n}{a'_p + d_n} + \sum_{c\in p, c\neq n}\frac{b_c d_c}{a_c + d_c}
\end{equation}
\begin{equation}
    c'_n = c'_p + \frac{{b'_p}^2}{a'_p + d_n} + \frac{\log(d_n) - \log(a'_p+d_n)}{2} +  \sum_{c\in p, c\neq n} c_c  + \frac{b_c^2}{d_c + a_c} + \frac{\log(d_c) - \log(a_c+d_c)}{2}
\end{equation}

The distribution of positions at an internal node is then
\begin{equation}
    \log P(x) = -a_n r_n^2 + 2 b_n r_n - a'_n d_n/(d_n + a'_n) r_n^2 + b'_n d_n/(d_n + a'_n) + C
\end{equation}
This can be rearranged into a standard Gaussian form
\begin{equation}
    \log P(x) = -(a_n + a'_n d_n/(d_n + a'_n)) (r_n - (b_n + b'_n d_n/(d_n + a'_n))/(a_n + a'_n d_n/(d_n + a'_n)))^2 + C
\end{equation}
such that the mean is
\begin{equation}
    x_n = b_n/a_n + b'_n d_n/(d_n + a'_n)
\end{equation}
and the variance ($() = 1/2\sigma^2$)
\begin{equation}
    \sigma^2_n = \frac{1}{2(a_n + a'_n d_n/(d_n + a'_n))} = \frac{(d_n + a'_n)}{2(a_n(d_n + a'_n) + a'_n d_n)}
\end{equation}

\section{common examples}

\subsection{WNV}
West-Nile virus was first detected on the east coast of the US in 1999 and reached the West Coast 5 years later in 2004. This translates to a wave front velocity of about 1000 km/year. Previous analysis estimated a diffusion coefficient of between 200 and 10000 km^2/day (Pybus et al). Using $v_{FKPP} = \sqrt{2D\alpha}$ and equating it to 1000 km/year = 3km/day, we obtain ranges of $\alpha=\frac{10 km^2}{2D day^2} = 0.025/day \ldots 0.0005/day$ or between $0.0035$ and $0.14$/week.
A pathogen that produces seasonal outbreaks with many orders of magnitude in prevalence between peak and troughs typically requires a growth rate of 1/week, considerably more than what the estimates of $D$ and the FKPP equation imply.
Alternatively, if we assume $\alpha = 0.1/day$, we would expect $D = \frac{10 km^2}{0.2 day} = 50 km^2/day$, implying a daily exploration radius of about 10km.
This picture is also consistent with the apparent ``slowing down" if lineages after the initial expansion across North America: Once the habitat was fully explored, directed invasion with $v_{FKPP}$ ceases.

\subsection{rabies}
Dispersal rates of rabies are typically estimated to be between 500 and 1500 km^2/year.
The most recent common ancestor of various populations is between 50 and 100 years in the past.
This would translate into rather limited spread of 200 to 500km of individual clades from their common ancestor, suggesting that these populations are highly fragmented and their long range dispersal is dominated by rare introductions of the virus into new populations.
Within each population, phylogeographic reconstruction will often give plausible results, but it is hard to extrapolate to longer time and distance scales.

\end{document}