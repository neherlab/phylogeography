\documentclass[aps,rmp, onecolumn]{revtex4}
%\documentclass[a4paper,10pt]{scrartcl}
%\documentclass[aps,rmp,twocolumn]{revtex4}

\usepackage[utf8]{inputenc}
\usepackage{amsmath,graphicx}
\usepackage{color}
%\usepackage{cite}

\newcommand{\bq}{\begin{equation}}
\newcommand{\eq}{\end{equation}}
\newcommand{\bn}{\begin{eqnarray}}
\newcommand{\en}{\end{eqnarray}}
\newcommand{\Richard}[1]{{\color{red}Richard: #1}}
\newcommand{\LH}{\mathcal{L}}

\begin{document}
\title{Inference of ancestral locations}
\author{Richard Neher}
\date{\today}
\maketitle
A common phylogeographic model is diffusion.
Given the sampling locations $r_i$ of the tips $i$ of the tree, the ancestral locations $r_n$ of internal nodes of the tree have a likelihood function
\begin{equation}
    \log L = -\sum_{n\neq root} \frac{(r_n - r_p)^2}{4Dt_n}
\end{equation}
where $r_p$ is the location of the parent of $n$ and $t_n$ is the length of the branch leading to node $n$.
The ancestral locations are not observed and to calculate the overall likelihood need to be integrated over.

This can be done recursively, where the likelihood of a subtree $n$ given the position of node $r_n$ is given by
\begin{equation}
    \begin{split}
        L_n(r_n) & = \prod_{c\in n}\frac{1}{4\pi D t_c}\int  d\, r_c e^{- \frac{(r_c - r_n)^2}{4Dt_c}}P(r_c) \\
        & = \prod_{c\in n} \frac{\sqrt{d_c}}{\sqrt{\pi}}\int d\, r_c e^{- d_c r_n^2 + 2d_c r_n r_c - d_c r_c^2 - a_c r_c^2 + 2 b_c r_c - c_c} \\
        & = \prod_{c\in n} \frac{\sqrt{d_c}}{\sqrt{\pi}}\int d\, r_c e^{- d_c r_n^2 + 2r_c(b_c + r_n d_c) - r_c^2 \left(d_c + a_c\right) - c_c} \\
        & = \prod_{c\in n} \frac{\sqrt{d_c}}{\sqrt{\pi}}\int d\, r_c e^{- d_c r_n^2 - (d_c + a_c)(r_c^2 - 2r_c\frac{b_c + d_c r_n}{d_c + a_c} + \frac{(b_c + d_c r_n)^2}{(d_c + a_c)^2}) + \frac{(b_c + d_c r_n)^2}{d_c + a_c} - c_c} \\
        & = \prod_{c\in n} \frac{\sqrt{d_c}}{\sqrt{\pi}}\int d\, r_c e^{- d_c r_n^2 + (d_c + a_c)(r_c - \frac{b_c + d_c r_n}{d_c + a_c})^2 + \frac{(b_c + d_c r_n)^2}{d_c + a_c} - c_c} \\
        & = \frac{\sqrt{d_c}}{\sqrt{a_c + d_c}} e^{- d_c r_n^2 + (b_c^2 + 2r_n b_c d_c + d_c^2 r_n^2 )/\left(d_c + a_c\right) - c_c} \\
        & = \frac{\sqrt{d_c}}{\sqrt{a_c + d_c}} e^{- d_c\left(1 - \frac{d_c}{a_c+d_c}\right) r_n^2 +  2\frac{b_c d_c}{d_c+a_c} r_n  - c_c + \frac{b_c^2}{d_c + a_c}} \\
    \end{split}
\end{equation}
This allows calculation of the parameters $a_n$, $b_n$, and $c_n$ of node $n$ from the children that are not terminal nodes as.
\begin{equation}
    a_n = \sum_{c\in n}d_c \left(1-\frac{d_c}{a_c+d_c}\right) = \sum_{c\in n} \frac{d_c a_c}{a_c+d_c}
\end{equation}
\begin{equation}
    b_n = \sum_{c\in n}\frac{b_c d_c}{a_c + d_c}
\end{equation}
\begin{equation}
    c_n = \sum_{c\in n} c_c  + \frac{b_c^2}{d_c + a_c} + \frac{\log(d_c) - \log(a_c+d_c)}{2}
\end{equation}
If a child is a terminal node, the terms in the sum need to be replaced by
\begin{equation}
    a_n = \sum_{c\in n} d_c
\end{equation}
\begin{equation}
    b_n = \sum_{c\in n} d_c r_c
\end{equation}
\begin{equation}
    c_n = \sum_{c\in n} d_c r_c^2 - \log(2\pi/d_c)/2
\end{equation}
Note that for a single child, the variances add ($a_n^{-1} = a_c^{-1} + d_c^{-1}$) and the most likely positions don't change ($b_n/a_n = b_c/a_c$).

The same propagation can be used up the tree
\begin{equation}
    a'_n  =  \frac{d_n a'_p}{a'_p+d_n} + \sum_{c\in p, c\neq n} \frac{d_c a_c}{a_c+d_c}
\end{equation}
\begin{equation}
    b'_n = \frac{b'_p d_n}{a'_p + d_n} + \sum_{c\in p, c\neq n}\frac{b_c d_c}{a_c + d_c}
\end{equation}
\begin{equation}
    c'_n = c'_p + + \frac{b'_p^2}{d_n + a'_p} + \frac{\log(d_p) - \log(a'_p+d_p)}{2} +  \sum_{c\in p, c\neq n} c_c  + \frac{b_c^2}{d_c + a_c} + \frac{\log(d_c) - \log(a_c+d_c)}{2}
\end{equation}

The distribution of positions at an internal node is then
\begin{equation}
    log P(x) = -a_n r_n^2 + 2 b_n r_n - a'_n d_n/(d_n + a'_n) r_n^2 + b'_n d_n/(d_n + a'_n) + C
\end{equation}
This can be rearranged into a standard Gaussian form
\begin{equation}
    log P(x) = -(a_n + a'_n d_n/(d_n + a'_n)) (r_n - (b_n + b'_n d_n/(d_n + a'_n))/(a_n + a'_n d_n/(d_n + a'_n)))^2 + C
\end{equation}
such that the mean is
\begin{equation}
    x_n = b_n/a_n + b'_n d_n/(d_n + a'_n)
\end{equation}
and the variance ($() = 1/2\sigma^2$)
\begin{equation}
    \sigma^2_n = \frac{1}{2(a_n + a'_n d_n/(d_n + a'_n))} = \frac{(d_n + a'_n)}{2(a_n(d_n + a'_n) + a'_n d_n)}
\end{equation}


\end{document}